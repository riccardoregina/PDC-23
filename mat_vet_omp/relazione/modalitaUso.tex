\subsection{Esecuzione su Cluster SCOPE}
Il programma deve essere eseguito mediante un job script contenente direttive pbs (portable batch system).

Un esempio di jobscript:
\begin{lstlisting}
    #!/bin/bash
    ##########################
    # #
    # The PBS directives #
    # #
    ##########################
    #PBS -q <queue>
    #PBS -l nodes=1:ppn=<core da utilizzare>
    #PBS -N <file>
    #PBS -o <file.out>
    #PBS -e <file.err>
    ##########################################
    # -q coda su cui va eseguito il job #
    # -l numero di nodi richiesti #
    # -N nome job(stesso del file pbs) #
    # -o, -e nome files contenente l’output #
    ##########################################
    
    echo ------------------------------------------------------
    echo 'Job is running on node:'
    cat $PBS_NODEFILE
    echo ------------------------------------------------------
    PBS_O_WORKDIR=$PBS_O_HOME/FILE
    echo ------------------------------------------------------
    echo PBS: qsub is running on $PBS_O_HOST
    echo PBS: originating queue is $PBS_O_QUEUE
    echo PBS: executing queue is $PBS_QUEUE
    echo PBS: working directory is $PBS_O_WORKDIR
    echo PBS: execution mode is $PBS_ENVIRONMENT
    echo PBS: job identifier is $PBS_JOBID
    echo PBS: job name is $PBS_JOBNAME
    echo PBS: node file is $PBS_NODEFILE
    echo PBS: current home directory is $PBS_O_HOME
    echo PBS: PATH = $PBS_O_PATH
    echo ------------------------------------------------------
    
    export OMP_NUM_THREADS=2
    export PSC_OMP_AFFINITY=TRUE
    echo "Compilo..."
    gcc -fopenmp -lgomp -o $PBS_O_WORKDIR/<file> $PBS_O_WORKDIR/<file.c>
    echo "Eseguo..."
    $PBS_O_WORKDIR/<file> [arguments]
    echo ------------------------------------------------------
\end{lstlisting}

Una volta scritto il job script, per eseguirlo, bisognerà eseguire da terminale il comando 
\begin{lstlisting}
qsub FILE.pbs
\end{lstlisting}

In caso l'esecuzione dello script vada a buon fine, verranno generati FILE.out e FILE.err, i file di output ed errore, rispettivamente.

Per andare a visualizzare tali file, bisognerà eseguire da terminale i comandi
\begin{lstlisting}
cat FILE.err
cat FILE.out
\end{lstlisting}

\subsection{Esecuzione su macchina personale}
Poiché le macchine d'uso comune sono oramai, per la maggior parte, multicore, é possibile sfruttare il programma parallelo in esame su di esse.
Per elaborare le direttive omp, bisogna aggiungere dei flag (-fopenmp -lgomp) in fase di compilazione del codice sorgente.
Va quindi eseguito da terminale il comando:
\begin{lstlisting}
    gcc -fopenmp -lgomp <file.c> [-o <file>]
\end{lstlisting}
E poi, per l'esecuzione:
\begin{lstlisting}
    ./<file>
\end{lstlisting}