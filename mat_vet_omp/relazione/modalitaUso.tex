Il programma deve essere eseguito mediante un job script contenente direttive pbs (portable batch system).

Un esempio di jobscript:
\begin{lstlisting}
INSERIRE SCRIPT PER OMP
\end{lstlisting}

Una volta scritto il job script, per eseguirlo, bisognerà eseguire da terminale il comando 
\begin{lstlisting}
qsub FILE.pbs
\end{lstlisting}

In caso l'esecuzione dello script vada a buon fine, verranno generati FILE.out e FILE.err, i file di output e errore rispettivamente.

Per andare a visualizzare tali file, bisognerà eseguire da terminale i comandi
\begin{lstlisting}
cat FILE.err
cat FILE.out
\end{lstlisting}