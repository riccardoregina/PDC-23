\subsection{Header file richiamati}
\begin{lstlisting}
#include <omp.h>
#include <stdio.h>
#include <stdlib.h>
\end{lstlisting}

\subsection{Main}
Nel blocco del main, ovvero il punto di ingresso del programma, viene in primis effettuata la dichiarazione della matrice $A$ e delle sue dimensione $m$, $n$, del vettore $x$ di dimensione $dim$.
Viene controllato se $dim$ e $n$ coincidano, ovvero se il prodotto è ben definito.
A questo punto, se il controllo è andato  a buon fine, viene chiamata la funzione centrale del programma.
Infine, viene eseguita la stampa del vettore $y$ risultante.

\begin{lstlisting}
int main(){
    int m, n;
    double **A = readMatrixFromFile("<path>", &m, &n);

    int dim;
    double *x = readVector("<path>", &dim);

    if(dim != n){
        printf("Vettore e matrice non hanno dimensioni adeguate per una moltiplicazione...");
        exit(-1);
    }
    
    double *y = matrixVector(m, n, x, A);
    printVector(y, m);

    return 0;
}
\end{lstlisting}

\subsection{Algoritmo}
La direttiva omp \textit{\#pragma} indica l'inizio di un blocco parallelo: essa genera un team di thread, sottoprocessi che condividono la stessa area di memoria. Pertanto, è importante specificare ai thread quali saranno le variabili condivise, accessibili a tutti e regolate da un semaforo di sistema, e quali private, sempre sovrascrivibili perché indipendenti. 

Per decidere manualmente la visibilitá delle variabili é stata usata la direttiva \textit{default(none)}
Le variabili condivise sono le dimensioni della matrice e dei vettori, che devono poter essere visualizzate in ogni istante della computazione da ciascun thread.
Le variabili private sono gli indici dei due cicli for innestati. Ogni thread compierà le proprie iterazioni indipendenti.
Per la distribuzione delle iterazioni del for-statement, si è usata una specifica direttiva  standard di omp: \textit{\#pragma...for...}. 
La funzione infine restituirà al main l'indirizzo di memoria del vettore risultante $y$.
\begin{lstlisting}
double *matrixVector(int m, int n, double *x, double **A){
    
    double *y = calloc(m, sizeof(double));
    int i, j;
    #pragma omp parallel for default(none) shared(m,n,x,A,y) private(i,j)
        for (i = 0; i < m; i++){
            for (j = 0; j < n; j++){
                y[i] += A[i][j] * x[j];
            }
        }
    return y;
}
\end{lstlisting}

\subsection{Funzioni accessorie}
Mostriamo adesso le altre funzioni utilizzate nel blocco del main.
\subsubsection{Lettura da file della matrice e del vettore}
\begin{lstlisting}
double **readMatrixFromFile(char *path, int *m, int *n){
    FILE *fp;
    fp = fopen(path,"r");
    int i,j;

    fscanf(fp, "%d %d", m, n);

    double **A = malloc((*m) * sizeof(double *));
    for (i = 0; i < (*m); i++)
        A[i] = malloc((*n) * sizeof(double));

    
    for (i = 0; i < (*m); i++){
        for (j = 0; j < (*n); j++){
            fscanf(fp, "%lf", &(A[i][j]));
        }
    }

    fclose(fp);
    return A;
}
\end{lstlisting}

\begin{lstlisting}
double *readVector(char *path, int *m){
    FILE *fp;
    fp = fopen(path,"r");
    int i;
    fscanf(fp, "%d", m);

    double *x = malloc((*m) * sizeof(double));

    for(i = 0; i < (*m); i++){
        fscanf(fp, "%lf", x+i);
    }
    fclose(fp);
    return x;
}
\end{lstlisting}

\subsubsection{Stampa del vettore risultante}
\begin{lstlisting}
void printVector(double *y,int m){
    int i;
    for(i = 0; i < m; i++){
        printf("%f ", y[i]);
    }
    printf("\n");
}
\end{lstlisting}