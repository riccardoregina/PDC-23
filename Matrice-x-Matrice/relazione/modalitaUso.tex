Il programma deve essere eseguito mediante un job script contenente direttive pbs (portable batch system).

Un esempio di jobscript:
\begin{lstlisting}
#!/bin/bash

#Sostituire opportunamente le espressioni tra parentesi angolari 
#(e.g. <NUMERO_DI_PROCESSORI>, <FILE> etc.) in ogni istanza

#PBS -q studenti
#PBS -l nodes= NUMERO_DI_PROCESSORI
#PBS -N <FILE>
#PBS -o <FILE>.out
#PBS -e <FILE>.err

# -q coda su cui va eseguito il job
# -l numero di nodi richiesti
# -N nome job(stesso del file pbs)
# -o, -e nome files contenente l’output

NCPU=`wc -l < $PBS_NODEFILE`
echo ----------------------------
echo 'This job is allocated on '${NCPU}' cpu(s)'
echo 'Job is running on node(s): '
cat $PBS_NODEFILE

PBS_O_WORKDIR=$PBS_O_HOME/<DIRECTORY_OF_FILE>

echo "Compilo..."
/usr/lib64/openmpi/1.4-gcc/bin/mpicc -o $PBS_O_WORKDIR/<FILE> $PBS_O_WORKDIR/<FILE>.c

echo "Eseguo..."
/usr/lib64/openmpi/1.4-gcc/bin/mpiexec -machinefile $PBS_NODEFILE -np $NCPU $PBS_O_WORKDIR/<FILE>
\end{lstlisting}

Una volta scritto il job script, per eseguirlo, bisognerà eseguire da terminale il comando 
\begin{lstlisting}
qsub <FILE>.pbs
\end{lstlisting}

In caso l'esecuzione dello script vada a buon fine, verranno generati FILE.out e FILE.err, i file di output e errore rispettivamente.

Per andare a visualizzare tali file, bisognerà eseguire da terminale i comandi
\begin{lstlisting}
cat <FILE>.err
cat <FILE>.out
\end{lstlisting}